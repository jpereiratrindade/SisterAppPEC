\documentclass[a4paper,12pt]{article}
\usepackage[utf8]{inputenc}
\usepackage[T1]{fontenc}
\usepackage[portuguese]{babel}
\usepackage{geometry}
\usepackage{amsmath}
\usepackage{graphicx}
\usepackage{hyperref}
\usepackage{listings}
\usepackage{xcolor}

\geometry{top=3cm, bottom=2cm, left=2.5cm, right=2.5cm}

\title{\textbf{SisterApp Engine v3.3.0}\\Manual Técnico de Modelos Computacionais}
\author{José Pedro Trindade}
\date{\today}

\begin{document}

\maketitle
\tableofcontents
\newpage

\section{Introdução}
O \textbf{SisterApp Engine} integra modelos computacionais dinâmicos para simular interações entre resiliência ecológica, capacidade produtiva e dinâmicas sociais em um ambiente voxel procedural. Este documento detalha as bases matemáticas e lógicas dos dois modelos principais: \textbf{Resiliência} e \textbf{Vegetação}.

\section{Modelo de Resiliência}
O Modelo de Resiliência é composto por três variáveis de estado globais, normalizadas entre $[0.0, 1.0]$. Estas variáveis alteram fundamentalmente os algoritmos de geração de terreno (ruído Perlin) e distribuição de blocos.

\subsection{1. Resiliência Ecológica ($R_{ecol}$)}
Representa a capacidade do sistema natural de absorver choques e manter sua estrutura.
\begin{itemize}
    \item \textbf{Impacto no Terreno:} Modula a frequência e amplitude do ruído de detalhe.
    \item \textbf{High $R_{ecol}$ ($>0.7$):} Terreno \textbf{complexo}, robusto e rico em variações de alta frequência, simulando um ecossistema saudável e biodiverso.
    \item \textbf{Low $R_{ecol}$ ($<0.3$):} Terreno \textbf{simplificado}, erodido e com baixa complexidade topológica, simulando degradação ambiental e perda de estruturas naturais.
\end{itemize}

\subsection{2. Resiliência Produtiva ($R_{prod}$)}
Refere-se à capacidade do sistema de gerar recursos (biomassa).
\begin{itemize}
    \item \textbf{Impacto na Vegetação:} Atua como multiplicador direto na densidade de árvores e fertilidade do solo.
    \item \textbf{Fórmula Simplificada:}
    \[
    Density_{tree} = Noise(x, z) \times R_{prod} \times Moisture
    \]
    \item \textbf{Baixa $R_{prod}$:} Resulta em escassez de recursos (poucas árvores), solos áridos (substituição de Grass por Dirt/Sand).
    \item \textbf{Alta $R_{prod}$:} Florestas densas e solos ricos.
\end{itemize}

\subsection{3. Resiliência Social ($R_{soc}$)}
Simula a organização humana e conectividade.
\begin{itemize}
    \item \textbf{Corredores Sociais:} $R_{soc}$ gera máscaras de "corredores" (estradas, clareiras) usando ruído de Voronoi ou Perlin de baixa frequência.
    \item \textbf{Conectividade:}
    \[
    Mask_{corridor} = |Noise_{lowFreq}(x, z)| < (0.1 \times R_{soc})
    \]
    \item \textbf{Visual:} $R_{soc}$ elevado cria caminhos planos e conectados através do terreno, facilitando navegação e simulando infraestrutura. Baixo $R_{soc}$ resulta em isolamento.
\end{itemize}

\section{Modelo de Vegetação}
O modelo de vegetação opera sobre o terreno gerado, determinando a cobertura do solo e a flora.

\subsection{Parâmetros Biofísicos}
Para cada coluna de voxels $(x, z)$, o sistema calcula:
\begin{enumerate}
    \item \textbf{Umidade ($M$):} Derivada de ruído Perlin + Proximidade da água.
    \item \textbf{Temperatura ($T$):} Inversamente proporcional à altitude ($Y$).
    \item \textbf{Fertilidade ($F$):} Combinação de $R_{prod}$, $M$ e tipo de solo.
\end{enumerate}

\subsection{Regras de Renderização}
\begin{itemize}
    \item \textbf{Grama (Grass):} Cresce em solos férteis. A cor é modulada pela Umidade (Verde vivo vs. Verde seco).
    \item \textbf{Árvores (Wood/Leaves):} Spawnam apenas em blocos de Grama onde $Density_{tree} > Threshold$.
    \item \textbf{Feedback Visual (Toggle):}
    \begin{itemize}
        \item \textbf{Vegetation ON:} Renderização normal.
        \item \textbf{Vegetation OFF:} Simula "Colapso": Grama é renderizada com cor de Terra (Dirt), Árvores são removidas.
    \end{itemize}
\end{itemize}

\section{Configuração do Usuário}
O usuário pode manipular estes modelos em tempo real via menu \textit{Tools $\to$ Performance}:
\begin{description}
    \item[Resilience Sliders:] Ajuste fino de $R_{ecol}$, $R_{prod}$, $R_{soc}$.
    \item[Vegetation Toggle:] Ativa/Desativa o sistema de vegetação para visualização de impacto.
    \item[Model Selection:] Alterna a topologia base (Plano, Suave, Ondulado).
\end{description}

\end{document}
