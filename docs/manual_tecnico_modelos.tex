\documentclass[a4paper,12pt]{article}
\usepackage[utf8]{inputenc}
\usepackage[T1]{fontenc}
\usepackage[portuguese]{babel}
\usepackage{geometry}
\usepackage{amsmath}
\usepackage{graphicx}
\usepackage{hyperref}
\usepackage{listings}
\usepackage{xcolor}

\geometry{top=3cm, bottom=2cm, left=2.5cm, right=2.5cm}

\title{\textbf{SisterApp Engine v3.4.0}\\Manual Técnico de Modelos Computacionais}
\author{José Pedro Trindade}
\date{\today}

\begin{document}

\maketitle
\tableofcontents
\newpage

\section{Introdução}
O \textbf{SisterApp Engine} integra sistemas avançados de análise topológica e persistência de dados. A versão 3.4.0 (com extensões v3.5) introduz um novo modelo focado na \textbf{Análise de Declividade (Slope Analysis)}, permitindo classificações precisas do terreno.

\section{Modelo de Análise de Declividade (Slope Analysis)}
Este modelo substitui a anterior lógica abstrata de resiliência por uma abordagem quantitativa baseada na inclinação local do terreno.

\subsection{1. Cálculo de Inclinação (Percentual)}
A declividade é calculada como a razão entre a diferença de altura (rise) e a distância horizontal (run), expressa em porcentagem. Para um ponto no terreno, a inclinação $S_{\%}$ é dada por:

\[
S_{\%} = \frac{\sqrt{(\Delta x)^2 + (\Delta z)^2}}{\text{run}} \times 100
\]
Onde:
\begin{itemize}
    \item $\Delta x$ e $\Delta z$ são os gradientes de altura nas direções X e Z.
    \item A distância base (\textit{run}) é definida pela resolução do voxel (2 unidades).
\end{itemize}

Isto permite uma correlação direta com normas técnicas de engenharia civil.

\subsection{2. Classificação Topológica (5 Classes)}
O terreno é segmentado em classes configuráveis pelo usuário. Os limiares (thresholds) padrão são:

\begin{table}[h]
\centering
\begin{tabular}{|l|l|l|}
\hline
\textbf{Classe} & \textbf{Intervalo ($S_{\%}$)} & \textbf{Descrição} \\ \hline
Flat (Plano) & $0\% - 3.0\%$ & Áreas adequadas para infraestrutura. \\ \hline
Gentle Slope (Suave) & $3.0\% - 8.0\%$ & Áreas de transição suave. \\ \hline
Rolling (Ondulado) & $8.0\% - 20.0\%$ & Terreno ondulado, requer terraplanagem. \\ \hline
Steep Slope (Íngreme/Forte) & $20.0\% - 45.0\%$ & Encostas fortes, risco de erosão. \\ \hline
Mountain (Montanha) & $> 45.0\%$ & Áreas inacessíveis ou de preservação. \\ \hline
\end{tabular}
\caption{Classes de Declividade Padrão (v3.5)}
\end{table}

\subsection{3. Persistência e Configuração}
Diferente dos modelos anteriores, todas as configurações de declividade são \textbf{persistentes}. O sistema serializa os limiares definidos pelo usuário em um arquivo JSON (`prefs.json`), garantindo que os critérios de análise sejam mantidos entre sessões.

\section{Modelo de Vegetação (Suspenso)}
Na versão 3.4.0, a geração de vegetação foi temporariamente suspensa para permitir foco total na validação das camadas de análise topológica. O sistema de tipos de solo (Grass, Dirt, Stone) permanece ativo para feedback visual.

\section{Geração de Topologia (Terrain Models)}
É fundamental distinguir o \textbf{Gerador de Topologia} do \textbf{Analisador de Declividade}. O sistema mantém três perfis de geração baseados em ruído Perlin, que definem a geometria física do mundo:

\begin{itemize}
    \item \textbf{Rippled Flat:} Baixa frequência base, gera predominantemente classes \textit{Flat} e \textit{Gentle}.
    \item \textbf{Smooth Hills:} Frequência média, introduz áreas \textit{Rolling}.
    \item \textbf{Rolling Hills:} Alta amplitude, necessária para gerar áreas \textit{Steep} e \textit{Mountain} para validação.
\end{itemize}

O fluxo de processamento é:
\[ \text{Modelo (Geometria)} \rightarrow \text{Voxel Grid} \rightarrow \text{Slope Analysis (Classificação)} \]

\section{Configuração do Usuário}
Interface atualizada no menu \textit{Tools}:
\begin{description}
    \item[Slope Sliders:] Ajuste dos limites percentuais para cada classe.
    \item[Probe Tool:] Ferramenta de diagnóstico (clique esquerdo) mostra $S_{\%}$ exato.
    \item[Persistence:] Botões para salvar/carregar preferências manualmente.
\end{description}

\section{Modelo de Drenagem (D8 Flow)}
A partir da versão v3.6.0, o sistema substituiu o modelo estocástico de erosão por partículas por um algoritmo determinístico de drenagem D8 (Steepest Descent).

\subsection{1. Direção do Fluxo (Flow Direction)}
Para cada célula do grid de terreno, o algoritmo determina a direção de escoamento para um dos 8 vizinhos com maior gradiente descendente.
\[
\text{Receiver} = \text{argmax}_{n \in \text{Neighbors}} (H_{\text{current}} - H_n)
\]
Se $H_{\text{current}} - H_n \leq 0$ para todos os vizinhos (mínimo local), a célula é um "sink" (sumidouro).

\subsection{2. Acumulação de Fluxo (Flow Accumulation)}
O fluxo é calculado iterativamente, ordenando as células por altura (decrescente). Cada célula transfere seu valor de fluxo acumulado para o seu vizinho receptor (Receiver), simulando a conservação de massa da água.
\[
F_{\text{receiver}} += F_{\text{upstream}}
\]
O resultado é um \textit{Flux Map} onde valores altos representam rios e canais principais.

\subsection{3. Visualização}
O shader utiliza o mapa de fluxo acumulado para renderizar recursos hídricos:
\begin{itemize}
    \item \textbf{Canais Principais:} Células com fluxo $F > 1.0$ (limite visual configurável) são coloridas em Cyan (0.0, 0.8, 1.0).
    \item \textbf{Continuidade:} O método D8 garante redes de drenagem dendríticas contínuas sem artefatos geométricos ("spots").
\end{itemize}

\end{document}
