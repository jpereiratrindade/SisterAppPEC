\documentclass[a4paper,12pt]{article}
\usepackage[utf8]{inputenc}
\usepackage[T1]{fontenc}
\usepackage[portuguese]{babel}
\usepackage{geometry}
\usepackage{amsmath}
\usepackage{graphicx}
\usepackage{hyperref}
\usepackage{listings}
\usepackage{xcolor}

\geometry{top=3cm, bottom=2cm, left=2.5cm, right=2.5cm}

% -----------------------------------------------------------------------
% CONFIGURAÇÕES DO HYPERREF
% -----------------------------------------------------------------------
\hypersetup{
	colorlinks=true,
	linkcolor=black,
	citecolor=black,
	urlcolor=blue
}

\title{\textbf{SisterApp: Plataforma de Ecologia Computacional v3.8.4}\\Manual Técnico de Modelos Computacionais}
\author{José Pedro Trindade}
\date{\today}

\begin{document}

\maketitle
\tableofcontents
\newpage

\section{Introdução}
O \textbf{SisterApp} evoluiu de uma engine gráfica para uma plataforma científica robusta focada em ecologia computacional. A versão 3.8.0 consolida ferramentas de navegação, análise de paisagem e validação métrica. A versão 3.8.4 remove definitivamente o suporte a Voxel para focar em terrenos de alta fidelidade (Finite World).

\section{Modelo de Análise de Declividade (Slope Analysis)}
Este modelo substitui a anterior lógica abstrata de resiliência por uma abordagem quantitativa baseada na inclinação local do terreno.

\subsection{Cálculo de Inclinação (Percentual)}
A declividade é calculada como a razão entre a diferença de altura (rise) e a distância horizontal (run), expressa em porcentagem. Para um ponto no terreno, a inclinação $S_{\%}$ é dada por:

\[
S_{\%} = \frac{\sqrt{(\Delta x)^2 + (\Delta z)^2}}{\text{run}} \times 100
\]
Onde:
\begin{itemize}
    \item $\Delta x$ e $\Delta z$ são os gradientes de altura nas direções X e Z.
    \item A distância base (\textit{run}) é definida pela resolução do voxel (2 unidades).
\end{itemize}

Isto permite uma correlação direta com normas técnicas de engenharia civil.

\subsection{Classificação Topológica (5 Classes)}
O terreno é segmentado em classes configuráveis pelo usuário. Os limiares (thresholds) padrão são:

\begin{table}[h]
\centering
\begin{tabular}{|l|l|l|}
\hline
\textbf{Classe} & \textbf{Intervalo ($S_{\%}$)} & \textbf{Descrição} \\ \hline
Flat (Plano) & $0\% - 3.0\%$ & Áreas adequadas para infraestrutura. \\ \hline
Gentle Slope (Suave) & $3.0\% - 8.0\%$ & Áreas de transição suave. \\ \hline
Rolling (Ondulado) & $8.0\% - 20.0\%$ & Terreno ondulado, requer terraplanagem. \\ \hline
Steep Slope (Íngreme/Forte) & $20.0\% - 45.0\%$ & Encostas fortes, risco de erosão. \\ \hline
Mountain (Montanha) & $> 45.0\%$ & Áreas inacessíveis ou de preservação. \\ \hline
\end{tabular}
\date{17 de Dezembro de 2025 - v3.8.4 (Finite World)}
\end{table}



\section{Modelo de Vegetação Campestre (Grassland Model)}
O SisterApp v3.9.1 introduz um modelo de dinâmica de vegetação campestre baseado em princípios de ecologia espacial e regimes de distúrbio (Fogo e Pastejo). Devido à complexidade biológica, este módulo possui uma **Documentação de Domínio (DDD) Exclusiva** que define suas regras e invariantes.

\subsection{Estrutura de Dois Estratos (DDD)}
A vegetação é modelada como dois estratos competitivos em cada célula do grid ($1m^2$):
\begin{itemize}
    \item \textbf{Estrato Inferior (EI):} Gramíneas e herbáceas. Alta taxa de crescimento, alta resiliência ao pastejo.
    \item \textbf{Estrato Superior (ES):} Subarbustos e arbustos baixos (geralmente $< 1m$). Crescimento lento e não pastejado. Apesar do porte baixo, acumula biomassa lenhosa que, quando senescente, atua como combustível principal.
\end{itemize}

O estado de cada célula é definido por um vetor de 4 componentes:
\[ V_{cell} = \{ Coverage_{EI}, Coverage_{ES}, Vigor_{EI}, Vigor_{ES} \} \]

\subsection{Definição de Escala e Resolução}
A célula ($V_{cell}$) é a unidade atômica da simulação, representando um estado homogêneo estatístico. Embora a simulação opere em espaço de grade abstrato (Grid Space), a interpretação física da "Cobertura" ($C$) é dependente da resolução métrica configurada na interface:

\subsubsection{Resolução Física ($R$)}
A dimensão espacial da célula é definida pelo parâmetro \textbf{Cell Size (Resolution)}, acessível na interface do usuário sob o menu "Map Generator". Este parâmetro permite o ajuste dinâmico em tempo de execução de $R \in [0.1m, 4.0m]$. O valor padrão é $R = 1.0m$. A área física de uma célula é dada por $A_{cell} = R^2$.

\subsubsection{Interpretação da Cobertura}
O valor de cobertura $C \in [0, 1]$ representa a fração da área da célula ocupada pelo estrato:
\[ \text{Área Ocupada} (m^2) = C \cdot R^2 \]

Exemplos:
\begin{itemize}
    \item Para $R=1.0m$ (Padrão): $C=0.5$ implica $0.5m^2$ de biomassa.
    \item Para $R=0.5m$ (Alta Definição): A área total é $0.25m^2$, logo $C=0.5$ implica $0.125m^2$.
\end{itemize}

Esta abstração permite que o modelo seja agnóstico à escala (Scale-Independent) em sua lógica interna, enquanto a renderização adapta a densidade visual à geometria física.

\subsubsection{Definição de Vigor ($\phi$)}
O Vigor ($\phi \in [0, 1]$) não representa biomassa, mas sim o **estado fisiológico** da planta.
\begin{itemize}
    \item $\phi \approx 1.0$: Turgor máximo, alta clorofila, baixo estresse hídrico. (Visual: Verde)
    \item $\phi < 0.5$: Senescência ou estresse severo. (Visual: Amarelo/Marrom)
    \item $\phi = 0.0$: Planta morta (Necromassa em pé).
\end{itemize}

Diferente de modelos estáticos, o Vigor no SisterApp é dinâmico e espacialmente heterogêneo:
\[ \phi_{target}(x,y) = 0.8 + 0.2 \cdot \text{Noise}(x,y) \]
A vegetação busca constantemente esse alvo, simulando microclimas variados. Se perturbada (ex: pisoteio ou seca não simulada explicitamente), o vigor decai, aumentando a probabilidade de fogo ($\mathcal{F}$) antes mesmo da perda de cobertura.

\subsection{Definição Operacional do Distúrbio}
O distúrbio ($D$) é definido como uma variável escalar composta que integra três dimensões fundamentais:
\begin{itemize}
    \item \textbf{Magnitude} ($M$): intensidade do evento;
    \item \textbf{Frequência Ecológica} ($F$): probabilidade relativa do regime (0.1 = Raro, 0.9 = Crônico);
    \item \textbf{Escala Espacial} ($E$): proporção da paisagem afetada.
\end{itemize}
\[ D = M \cdot F \cdot E \]

\subsection{Resposta Funcional ao Distúrbio}
O sistema modela a capacidade de suporte ($K$) de cada estrato como uma função direta do regime de distúrbio vigente:

\subsubsection{Estrato Inferior (EI - Gramíneas)}
Apresenta resposta \textbf{logarítmica positiva}:
\[ R_{EI}(D) = \text{clamp}(\log(1 + \alpha D), 0, 1) \]
Onde $\alpha$ é o coeficiente de sensibilidade (Ganho). Ecologicamente, isso expressa que distúrbios de baixa intensidade já promovem forte resposta do EI, saturando em regimes intensos.

\subsubsection{Estrato Superior (ES - Arbustos)}
Apresenta resposta \textbf{exponencial negativa}:
\[ R_{ES}(D) = \text{clamp}(e^{-\beta D}, 0, 1) \]
Onde $\beta$ é o coeficiente de decaimento. Isso captura a supressão acelerada de lenhosas sob regimes de perturbação frequente.

\subsection{Modulação da Capacidade de Suporte}
Ao contrário do modelo anterior de remoção direta, os índices $R_{EI}$ e $R_{ES}$ modulam os alvos de equilíbrio:
\[ K_{EI}^{target} \propto R_{EI}(D) \quad \text{e} \quad K_{ES}^{target} \propto R_{ES}(D) \]
A vegetação então "relaxa" em direção a esses alvos ao longo do tempo ($\tau_{rec}$), permitindo transições suaves entre estados de savana, campo limpo e arbustal.

\subsection{Heterogeneidade Espacial (Spatial Noise)}
Para evitar a monotonia visual e simular a variabilidade edáfica não mapeada, a inicialização e a capacidade de suporte local ($K$) são moduladas por funções de ruído procedural (Perlin Noise). Isso gera manchas naturais de alta e baixa densidade, independentemente dos distúrbios.

\subsection{Visualização em Tempo Real}
O shader de terreno foi atualizado para combinar a coloração do solo (Pedologia) com a camada de vegetação.
\begin{itemize}
    \item \textbf{Realistic Mode:} Mistura texturas de solo e vegetação baseada na cobertura. O Vigor modula a cor entre Verde (Saudável) e Amarelo/Marrom (Senescente).
    \item \textbf{Heatmaps:} Modos de diagnóstico para visualizar cobertura bruta de EI/ES e níveis de estresse fisiológico (Vigor).
\end{itemize}

\subsection{Inovações de Visualização Científica (Roadmap)}
Para aumentar a fidelidade ontológica da plataforma, foram definidos conceitos avançados de tradução visual (DDD Visual):

\subsubsection{Síntese de NDVI Virtual}
Para validação cruzada com sensoriamento remoto, o sistema prevê a geração de "Falso-Cor" baseada na reflectância espectral simulada:
\[ NDVI_{sim} = \frac{(C_{total} \cdot Vigor) - (1 - Vigor)}{(C_{total} \cdot Vigor) + (1 - Vigor)} \]
Isso permite comparar diretamente os output do modelo com imagens Sentinel-2 ou Landsat.

\subsubsection{Sinalização de Estresse (Early Warning)}
Diferente de engines de jogos que apenas "removem" a planta morta, o SisterApp implementa a \textbf{Atenuação Visual}: plantas sob estresse hídrico ou de pastejo manifestam redução de turgor (suavização de normal map) e transparência antes da perda efetiva de biomassa.

\subsection{Formulação Matemática Detalhada}
O modelo utiliza uma abordagem híbrida de Autômatos Celulares e Equações Diferenciais Discretas. O estado de cada célula $i$ no tempo $t$ é vetorizado como:
\[ \mathbf{V}_i(t) = [ C_{EI}, C_{ES}, \phi_{EI}, \phi_{ES}, \tau_{rec} ] \]
Onde $C$ é a cobertura (0-1), $\phi$ é o vigor fisiológico (0-1) e $\tau_{rec}$ é o temporizador de recuperação pós-distúrbio.

\subsubsection{Dinâmica de Crescimento e Recuperação}
Quando $\tau_{rec} \le 0$, a vegetação recupera biomassa seguindo uma função logística linearizada próxima à capacidade de suporte $K$:

\[
\frac{dC_{EI}}{dt} = r_{EI} \cdot (K_{EI} - C_{EI}) \cdot \phi_{EI}
\]
Implementado numericamente como:
\begin{lstlisting}[language=C++]
if (grid.ei_coverage[i] < maxEI) {
    grid.ei_coverage[i] += 0.1f * dt; // Taxa r = 0.1
    if (grid.ei_coverage[i] > maxEI) grid.ei_coverage[i] = maxEI;
}
\end{lstlisting}

A colonização por arbustos ($C_{ES}$) é dependente da presença prévia de gramíneas ($C_{EI} > 0.7$), simulando a sucessão secundária facilitada.

\paragraph{Determinismo e Estocasticidade}
Para garantir a reprodutibilidade científica dos experimentos, a geração de números aleatórios para eventos estocásticos utiliza um gerador \texttt{Mersenne Twister} (std::mt19937) com semente fixa, eliminando a variabilidade não-controlada entre execuções.

\subsubsection{Probabilidade de Fogo (Fire Probability)}
A ignição estocástica é calculada localmente com base na inflamabilidade do combustível disponível. O modelo adota a premissa ecológica de que **arbustos secos** ($ES_{dry}$) são o principal vetor de propagação:

1. **Fator de Secura ($\delta_{ES}$):**
\[ \delta_{ES} = \max(0, 1.0 - \phi_{ES}) \]

2. **Inflamabilidade ($\mathcal{F}_i$):**
\[ \mathcal{F}_i = \underbrace{C_{ES} \cdot (2\delta_{ES})}_{\text{Lenhoso Seco}} + \underbrace{C_{EI} \cdot 0.3 \cdot (1-\phi_{EI})}_{\text{Fino Morto}} \]
*Nota: A contribuição do ES é nula se $\delta_{ES} < 0.5$ (Vigor > 0.5).*

3. **Probabilidade de Ignição ($P_{ign}$):**
\[ P_{ign} = P_{base} + \alpha \cdot \mathcal{F}_i \]
Onde $P_{base} = 0.05$ e $\alpha = 0.8$. Se um número aleatório $R \in [0,1] < P_{ign}$, ocorre a remoção total de biomassa.

\subsubsection{Dinâmica de Pastejo}
O pastejo atua como uma força de remoção seletiva sobre o Estrato Inferior:
\[ C_{EI}(t+\Delta t) = C_{EI}(t) - I_{grazing} \cdot \Delta t \]
Com impacto colateral no vigor:
\[ \phi_{EI}(t+\Delta t) = \phi_{EI}(t) - \frac{I_{grazing}}{2} \cdot \Delta t \]
Isso simula o "superpastejo" que reduz não apenas a biomassa, mas a capacidade fotossintética futura.

\subsubsection{Implementação Computacional}
Para garantir a escalabilidade em grandes paisagens ($> 16 \text{ milhões de células}$), o sistema adota uma abordagem de \textit{Fixed-Time Step Simulation}:
\begin{itemize}
    \item \textbf{Frequency:} 5-10 Hz (desacoplado do Frame Rate de renderização).
    \item \textbf{Throttling:} A atualização de estado e a transferência de dados (CPU $\to$ GPU) são limitadas temporalmente para evitar gargalos no barramento PCIe.
\end{itemize}

\section{Geração de Topologia (Terrain Models)}
É fundamental distinguir o \textbf{Gerador de Topologia} do \textbf{Analisador de Declividade}. O sistema mantém três perfis de geração baseados em ruído Perlin, que definem a geometria física do mundo.

\subsection{Modelo Experimental Blend (v3.8.3)}
Este modelo introduz a composição ponderada de frequências de ruído, permitindo controle fino sobre a morfologia do terreno.
\[ H(x,z) = \text{Norm} \left( \sum_{i \in \{L, M, H\}} W_i \cdot Noise(x \cdot f_i, z \cdot f_i) \right)^{\gamma} \]
Onde:
\begin{itemize}
    \item $W_L, W_M, W_H$ são os pesos para frequências Baixa, Média e Alta.
    \item $\gamma$ é o expoente de nitidez (Sharpness).
    \item \textbf{Normalização:} Para evitar o achatamento (clipping) observado em versões anteriores, a soma ponderada é normalizada pelo total dos pesos ($\sum W_i$) antes de ser mapeada para a altura final.
\end{itemize}

\subsection{Perfis Padrão}
\begin{itemize}
    \item \textbf{Rippled Flat:} Baixa frequência base, gera predominantemente classes \textit{Flat} e \textit{Gentle}.
    \item \textbf{Smooth Hills:} Frequência média, introduz áreas \textit{Rolling}.
    \item \textbf{Rolling Hills:} Alta amplitude, necessária para gerar áreas \textit{Steep} e \textit{Mountain} para validação.
\end{itemize}

O fluxo de processamento é:
\[ \text{Modelo (Geometria)} \rightarrow \text{Heightmap Grid} \rightarrow \text{Slope Analysis (Classificação)} \]

\section{Configuração do Usuário}
Interface atualizada no menu \textit{Tools}:
\begin{description}
    \item[Slope Sliders:] Ajuste dos limites percentuais para cada classe.
    \item[Probe Tool:] Ferramenta de diagnóstico (clique esquerdo) mostra $S_{\%}$ exato.

\end{description}

\subsection{Variáveis de Controle Espacial}
O sistema permite o ajuste fino da topografia através de três variáveis principais:
\begin{enumerate}
    \item \textbf{Feature Size (Frequência):} Controla o tamanho horizontal das montanhas. Valores menores geram grandes maciços; valores maiores geram colinas frequentes.
    \item \textbf{Roughness (Persistência):} Controla a irregularidade da superfície. 
    \begin{itemize}
        \item Baixa ($<0.5$): Colinas suaves e dunas.
        \item Alta ($>0.5$): Terreno rochoso, escarpado e ruidoso.
    \end{itemize}
    \item \textbf{Amplitude:} A altura máxima vertical em metros.
    \item \textbf{Cell Size (Resolução):} A dimensão física de cada pixel da grade (em metros).
\end{enumerate}

\section{Modelo de Drenagem (D8 Flow)}
A partir da versão v3.6.0, o sistema substituiu o modelo estocástico de erosão por partículas por um algoritmo determinístico de drenagem D8 (Steepest Descent).

\subsection{Direção do Fluxo (Flow Direction)}
Para cada célula do grid de terreno, o algoritmo determina a direção de escoamento para um dos 8 vizinhos com maior \textbf{declividade} descendente (Steepest Slope).
\[
\text{Receiver} = \text{argmax}_{n \in \text{Neighbors}} \left( \frac{H_{\text{current}} - H_n}{\text{Distance}_n} \right)
\]
Onde $\text{Distance}_n$ é a distância física até o vizinho (Resolution para cardeais, $\text{Resolution} \times \sqrt{2}$ para diagonais).
Se o declive for $\leq 0$ para todos os vizinhos (mínimo local), a célula é um "sink" (sumidouro).

\subsection{Acumulação de Fluxo (Flow Accumulation)}
O fluxo é calculado iterativamente, ordenando as células por altura (decrescente). Cada célula transfere seu valor de fluxo acumulado para o seu vizinho receptor (Receiver), simulando a conservação de massa da água.
\[
F_{\text{receiver}} += F_{\text{upstream}}
\]
O resultado é um \textit{Flux Map} onde valores altos representam rios e canais principais.

\subsection{Visualização}
O shader utiliza o mapa de fluxo acumulado para renderizar recursos hídricos:
\begin{itemize}
    \item \textbf{Canais Principais:} Células com fluxo $F > 1.0$ (limite visual configurável) são coloridas em Cyan (0.0, 0.8, 1.0).
    \item \textbf{Continuidade:} O método D8 garante redes de drenagem dendríticas contínuas sem artefatos geométricos ("spots").
\end{itemize}

\section{Análise de Bacias Hidrográficas (Watershed Analysis)}
Introduzido na versão v3.6.3, este módulo permite a identificação e delimitação de bacias de drenagem baseadas na topologia D8.

\subsection{Segmentação Global}
O algoritmo de segmentação particiona todo o terreno em bacias distintas. O processo ocorre em duas etapas:
\begin{enumerate}
    \item \textbf{Identificação de Sinks}: Localização de todos os "sumidouros" (minimos locais ou bordas do mapa). Cada sink recebe um ID único.
    \item \textbf{Propagação Upstream (BFS)}: Um algoritmo de busca em largura (Breadth-First Search) percorre a rede de fluxo no sentido inverso (de jusante para montante), atribuindo o ID do sink a todas as células constituintes de sua área de contribuição.
\end{enumerate}

\subsection{2. Delineação Interativa}
Permite ao usuário consultar a bacia de contribuição de um ponto arbitrário $P(x,y)$. O sistema rastreia recursivamente todos os vizinhos que fluem para $P$, gerando uma máscara binária instantânea da área de captação a montante.

\subsection{3. Visualização de Contornos}
O usuário pode habilitar a opção "Show Contours" na interface. O sistema utiliza a derivada parcial do ID da bacia (via shader \texttt{fwidth}) para detectar arestas onde o ID muda, desenhando uma linha escura de 1 pixel sobre os limites das bacias para melhor distinção visual.

\section{Métricas Eco-Hidrológicas}
O Relatório Hidrológico foi expandido para incluir indicadores funcionais derivados da topografia:

\subsection{Índice Topográfico de Umidade (TWI)}
\[ TWI = \ln \left( \frac{A}{\tan \beta} \right) \]
Onde $A$ é a área de contribuição específica (fluxo) e $\tan \beta$ é a declividade local. O TWI estima zonas de saturação do solo. O sistema reporta a porcentagem da área com $TWI > 8.0$ como proxy para zonas úmidas.

\subsection{Densidade de Drenagem ($D_d$)}
\[ D_d = \frac{L_{total}}{Area_{total}} \]
Calculado como a razão entre células classificadas como "rio" (Fluxo > 100) e o total de células. Indica a permeabilidade e dissecação do relevo.

\subsection{Estatísticas por Bacia (Basin-Level Metrics)}
O sistema agora agrega métricas de elevação, declividade, TWI e densidade de drenagem individualmente para as 3 maiores bacias identificadas, permitindo uma análise comparativa da resposta hidrológica de diferentes sub-regiões do modelo.

\subsection{Geração Assíncrona (v3.8.3)}
A partir da versão 3.8.3, a geração de terrenos (especialmente em resoluções altas como $4096 \times 4096$) é executada de forma assíncrona em uma thread separada. Isso previne o congelamento da interface ("Not Responding") durante o processamento de milhões de células. Uma tela de carregamento informa o progresso ao usuário.

\section{Resolução Espacial Variável (V3.6.5)}
Para atender à necessidade de maior definição nos limites de bacias e redes de drenagem, foi introduzido o controle de \textbf{Cell Size (Resolução)}.

\subsection{Definição de Escala}
O usuário pode ajustar o tamanho métrico de cada célula (pixel) da grade de simulação:
\begin{itemize}
    \item \textbf{1.0 m (Padrão):} Equilíbrio entre cobertura de área e detalhe.
    \item \textbf{$< 1.0$ m (Alta Resolução):} Aumenta a densidade de vértices por unidade de área. Ideal para suavizar limites de bacias e detalhar canais de drenagem, reduzindo o efeito de "pixelização" (aliasing geométrico).
    \item \textbf{$> 1.0$ m (Baixa Resolução):} Permite cobrir grandes extents geográficos com menor custo computacional.
\end{itemize}

O sistema ajusta automaticamente a visualização e a lógica de interação (raycasting) para manter a coerência espacial independentemente da escala escolhida.

\section{Módulo de Análise de Solos (V3.7.3)}
O sistema inclui agora uma camada de pedologia probabilística baseada na declividade, conforme a tabela de relação Relevo-Solo definida pelo usuário.

\subsection{Metodologia: Ruído Coerente e Métricas de Paisagem (v3.8.0).}
\subsection{Minimap e Navegação Interativa}
A versão 3.8.0 introduz um Minimapa e controles de câmera aprimorados para facilitar a navegação e a compreensão espacial.
\begin{itemize}
    \item \textbf{Visualização Top-Down}: Renderiza o mapa de solos e relevo com neblina de guerra (Fog of War) simulada pela distância.
    \item \textbf{Símbolos (Alegorias)}: Um algoritmo de detecção de picos identifica máximos locais na topografia e desenha pequenos triângulos brancos, fornecendo referências visuais "game-like" para orientação.
    \item \textbf{Nível da Água (Water Level)}: Visualização configurável de zonas submersas (Azul), permitindo identificar depressões e lagos mesmo antes da simulação hidrológica.
    \item \textbf{Controles}:
    \begin{itemize}
        \item \textbf{Zoom}: Roda do mouse ajusta o Campo de Visão (FOV) no modo voo livre.
        \item \textbf{Minimap Zoom/Pan}: Roda do mouse e botão do meio dentro da janela do minimapa.
        \item \textbf{Teleporte}: Clique com botão esquerdo no minimapa para viagem rápida.
    \end{itemize}
\end{itemize}
Para reproduzir os padrões espaciais descritos pelos índices de Ecologia da Paisagem (LSI, CF, RCC), o sistema substituiu a distribuição aleatória simples por um algoritmo de \textbf{Competição de Padrões} baseado em ruído procedural (Perlin/Simplex).

Cada tipo de solo possui um "perfil de ruído" configurado para mimetizar suas métricas (Ver Tabela \ref{tab:manchas_solo_farina}):
\begin{itemize}
    \item \textbf{Domain Warping (Distorção):} Simula o LSI. Solos com alto LSI sofrem forte distorção de coordenadas, criando bordas complexas.
    \item \textbf{Frequência e Rugosidade:} Simulam o CF. Solos com alto CF utilizam mais oitavas de ruído fractal.
    \item \textbf{Anisotropia (Estiramento):} Simula o RCC. Solos com baixo RCC são esticados em um eixo para criar formas alongadas.
\end{itemize}

O solo final em cada pixel é determinado por uma competição onde o tipo com maior "força" de padrão local vence (dentre os candidatos válidos para a declividade).

\subsection{Classes e Cores}
\begin{itemize}
    \item \textbf{Plano (0-3\%):} Hidromórfico (Teal), B Textural (Laranja), Argila Expansiva (Roxo).
    \item \textbf{Suave (3-8\%):} B Textural, Bem Desenvolvido (Terracota), Argila Expansiva.
    \item \textbf{Ondulado (8-20\%):} B Textural, Argila Expansiva.
    \item \textbf{Forte (20-45\%):} B Textural, Solo Raso (Amarelo).
    \item \textbf{Montanhoso (45-75\%):} Solo Raso (Amarelo).
    \item \textbf{Escarpado ($>75\%$):} Afloramento Rochoso (Cinza).
\end{itemize}


\begin{table}[htbp]
	\centering
	\caption{Descritores de estrutura espacial das manchas de solo segundo métricas da Ecologia da Paisagem (Farina)}
	\label{tab:manchas_solo_farina}
	\begin{tabular}{lccc}
		\hline
		\textbf{Tipo de Solo} & 
		\textbf{LSI} & 
		\textbf{CF} & 
		\textbf{RCC} \\
		\hline
		Solo Raso & 
		5434.91 & 
		2.49 & 
		0.66 \\
		
		Bem Desenvolvido & 
		2508.07 & 
		2.36 & 
		0.68 \\
		
		Hidromórfico & 
		3272.30 & 
		2.27 & 
		0.65 \\
		
		Argila Expansiva & 
		1827.24 & 
		2.84 & 
		0.64 \\
		
		B--Textural & 
		2766.09 & 
		3.36 & 
		0.66 \\
		\hline
	\end{tabular}
	\begin{flushleft}
		\footnotesize
		\textbf{Nota:} 
		LSI = Índice de Forma da Paisagem (\textit{Landscape Shape Index}), indicador da complexidade geométrica média das manchas; 
		CF = Complexidade da Forma, expressando irregularidade e alongamento das manchas; 
		RCC = Coeficiente de Circularidade Relativa, variando de 0 (formas alongadas ou irregulares) a 1 (formas altamente compactas).
	\end{flushleft}
\end{table}

\subsection*{Descritores de forma das manchas}

A estrutura espacial das manchas de solo foi caracterizada por descritores clássicos da Ecologia da Paisagem, conforme proposto por Farina (1998, 2006), os quais permitem avaliar a complexidade geométrica, a irregularidade das bordas e o grau de compactação das manchas na paisagem. Foram utilizados o Índice de Forma da Paisagem (LSI), a Complexidade da Forma (CF) e o Coeficiente de Circularidade Relativa (RCC), descritos a seguir.

\paragraph{Índice de Forma da Paisagem (LSI)}
O LSI expressa a complexidade geométrica das manchas a partir da relação entre perímetro e área, sendo definido por:
\[
LSI = \frac{P}{2\sqrt{\pi A}}
\]
em que $P$ corresponde ao perímetro da mancha e $A$ à sua área. Valores de LSI próximos de 1 indicam manchas com formas simples e compactas, enquanto valores mais elevados refletem maior irregularidade e desenvolvimento de bordas.

\paragraph{Complexidade da Forma (CF)}
A Complexidade da Forma representa o grau de irregularidade geométrica das manchas, considerando o aumento relativo do perímetro em relação à área:
\[
CF = \frac{P}{A}
\]
em que $P$ é o perímetro e $A$ a área da mancha. Valores mais elevados de CF indicam formas mais alongadas, recortadas ou dendríticas.

\paragraph{Coeficiente de Circularidade Relativa (RCC)}
O Coeficiente de Circularidade Relativa avalia o quão próxima a forma da mancha está de um círculo perfeito, sendo calculado por:
\[
RCC = \frac{4\pi A}{P^{2}}
\]
em que $A$ corresponde à área da mancha e $P$ ao seu perímetro. O RCC varia entre 0 e 1, sendo valores próximos de 1 indicativos de manchas altamente compactas e valores menores associados a formas alongadas ou fragmentadas.



\section{Modelo Integrado Ecofuncional da Paisagem (v4.0)}
Com a atualização v4.0 (Dezembro/2025), o SisterApp transcende a simulação isolada de vegetação para incorporar um Modelo Integrado de Paisagem (Integrated Landscape Model - ILM), atendendo aos requisitos de acoplamento ecofuncional definidos na Documentação de Domínio (DDD).

\subsection{Definição do Domínio}
O domínio é definido como a modelagem integrada da paisagem campestre como um sistema ecofuncional complexo, espacialmente acoplado e temporalmente explícito. Nenhuma propriedade sistêmica é modelada diretamente; todas emergem da interação entre estados, processos e regimes de distúrbio.

\subsection{Arquitetura e Agregado Central (LandscapeCell)}
A \textbf{LandscapeCell} representa a unidade lógica mínima da paisagem. Embora implementada computacionalmente com estruturas "Structure of Arrays" (SoA) para eficiência de cache, conceitualmente ela agrega:
\begin{enumerate}
    \item \textbf{Identidade:} CellID único na grade.
    \item \textbf{Responsabilidades:} Representar um ponto espacial, referenciar estados biológicos/físicos e servir de base para métricas emergentes.
\end{enumerate}

\subsection{Estados de Domínio}
O estado de cada célula é composto por vetores independentes que interagem via serviços:

\subsubsection{SoilState (Fundação Edáfica)}
\begin{itemize}
    \item \textbf{SoilDepth ($d_{soil}$):} Variável crítica sujeita a erosão. Solos $< 0.2m$ limitam o vigor.
    \item \textbf{PropaguleBank ($\Pi$):} Memória ecológica que define a resiliência ($\tau_{rec}$).
    \item \textbf{InfiltrationCapacity:} Modulada pela matéria orgânica e compactação.
    \item \textbf{SurfaceExposure:} Fração de solo descoberto, aumentando risco de erosão.
\end{itemize}

\subsubsection{VegetationState (Biomassa)}
\begin{itemize}
    \item \textbf{Coverage ($C_{EI}, C_{ES}$):} Cobertura dos estratos Inferior e Superior.
    \item \textbf{Vigor ($\phi_{EI}, \phi_{ES}$):} Estado fisiológico (0-1).
    \item \textbf{StructuralHeterogeneity:} Variabilidade espacial local.
\end{itemize}

\subsubsection{HydrologicalState (Fluxo Ativo)}
\begin{itemize}
    \item \textbf{RunoffPotential:} Água disponível para escoamento superficial ($P - I_{infil}$).
    \item \textbf{FlowAccumulation:} Volume acumulado via D8.
    \item \textbf{ErosionRisk:} Potencial de destacamento de solo ($StreamPower \times (1 - Coverage)$).
\end{itemize}

\subsection{Serviços de Domínio (Processos Ecofuncionais)}
Os processos não pertencem à célula, mas operam sobre a grade como serviços:
\begin{itemize}
    \item \textbf{GrowthService:} Calcula crescimento logístico e competição inter-estratos.
    \item \textbf{RunoffService:} Resolve o fluxo hidrológico D8 a cada tick, transportando água e sedimentos.
    \item \textbf{ErosionService:} Remove $d_{soil}$ baseado no risco de erosão e deposita em bacias de sedimentação.
    \item \textbf{DisturbanceService:} Aplica regimes de Fogo e Pastejo conforme magnitude e frequência.
\end{itemize}

\subsection{Loop de Atualização e Tempo}
O sistema opera em tempo discreto ($\Delta t$). O ciclo de atualização conceitual segue a ordem:
\begin{enumerate}
    \item Aplicação de Regimes de Distúrbio.
    \item Execução de Processos Físicos (Hidrologia, Erosão).
    \item Execução de Processos Biológicos (Crescimento, Mortalidade).
    \item Atualização de Estados.
    \item Derivação de Propriedades Emergentes (ex: Resiliência).
\end{enumerate}

\subsection{Propriedades Emergentes e Invariantes}
\textbf{Resiliência Ecofuncional:} Não é uma variável, mas uma função da capacidade do sistema de recuperar $C_{veg}$ após $D$ (Distúrbio), dependendo de $\Pi$ (Banco de Propágulos) e $d_{soil}$.

\textbf{Invariantes do Sistema:}
\begin{itemize}
    \item Solo descoberto elevado $\implies$ Aumento não-linear de erosão.
    \item A perda funcional (Vigor) precede o colapso estrutural (Cobertura).
    \item Regeneração é impossível sem banco de propágulos (Estado de Desertificação).
\end{itemize}

\subsection{Interface de Controle e Diagnóstico (v4.0)}
Para suportar a validação deste modelo complexo, a interface foi expandida:
\begin{itemize}
    \item \textbf{Rain Intensity Slider:} Permite forçar tempestades extremas para testar $RunoffService$.
    \item \textbf{Ecological Probe:} A ferramenta de inspeção agora revela estados internos ocultos ($d_{soil}$, $OM$, $E_{risk}$) para "debug" biológico.
\end{itemize}

\subsection{Integração Machine Learning Genérica (v4.2.0)}
O sistema incorpora um módulo avançado de Machine Learning (\texttt{MLService}) capaz de gerenciar múltiplos modelos preditivos simultaneamente.

\subsubsection{Arquitetura Genérica}
A classe \texttt{MLService} foi refatorada para abstrair a natureza dos inputs e outputs, permitindo o suporte a qualquer modelo Perceptron via um mapa associativo:
\[ \text{Models} : \{ \text{"soil\_color"} \to P_1, \text{"hydro\_runoff"} \to P_2, \dots \} \]

\subsubsection{Pipeline de Treinamento}
O treinamento ocorre de forma assíncrona (background thread), utilizando vetores de entrada genéricos ($\mathbf{x} \in \mathbb{R}^n$) e alvos escalares ($y \in [0,1]$). Isso permite que novos modelos (ex: previsão de biomassa ou risco de fogo) sejam adicionados sem recompilação do núcleo da engine.

\subsubsection{Arquitetura Perceptron (Eigen)}
A implementação utiliza um Perceptron Multicamadas otimizado com a biblioteca Eigen para vetorização SIMD (AVX/SSE).
\\\\
\textbf{Modelo Matemático:}
\[ y = \sigma(\mathbf{W} \cdot \mathbf{x} + b) \]
Onde $\sigma(z) = \frac{1}{1 + e^{-z}}$ é a função de ativação Sigmoid. O modelo processa vetores de entrada normalizados das propriedades do solo:
\[ \mathbf{x} = [ \text{Depth}, \text{OM}, \text{Infiltration}, \text{Compaction} ] \quad (\text{para soil\_color}) \]

\subsubsection{Modelo de Hidrologia ("hydro\_runoff")}
Este modelo estima a geração de escoamento superficial sem resolver a equação de fluxo completa, permitindo inferências rápidas para visualização ou heurísticas de IA.
\\\\
\textbf{Inputs ($\mathbf{x}$):}
\begin{itemize}
    \item Intensidade da Chuva (Normalizada $0-100 mm/h$)
    \item Taxa de Infiltração Efetiva (Baseada no Tipo de Solo)
    \item Biomassa Vegetal ($C_{EI} + C_{ES}$)
\end{itemize}
\textbf{Target ($y$):}
\begin{itemize}
    \item Coeficiente de Runoff ou Fluxo Absoluto ($mm/h$).
\end{itemize}
\textbf{Geração de Dados Estocástica:}
Para garantir generalização, o treinamento não utiliza apenas a chuva atual do mapa. O sistema sorteia cenários de chuva aleatórios ($R \sim U[0, 100]$) para cada célula amostrada, ensinando a rede a prever o comportamento hidrológico sob diversas condições climáticas.

\subsubsection{Pipeline de Renderização Híbrida}
Diferente de abordagens "Black Box" puras, o SisterApp utiliza o ML como um "Surrogate Model" para enriquecer a visualização:
\begin{enumerate}
    \item A simulação física calcula os estados brutos ($d_{soil}$, etc).
    \item O \texttt{MLService} infere uma "Assinatura Espectral" (Cor) baseada nesses estados.
    \item O renderizador mistura essa predição com a textura base, permitindo visualizar correlações não-lineares aprendidas de dados reais (quando o modelo for treinado).
\end{enumerate}
