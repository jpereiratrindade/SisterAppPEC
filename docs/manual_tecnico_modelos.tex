\documentclass[a4paper,12pt]{article}
\usepackage[utf8]{inputenc}
\usepackage[T1]{fontenc}
\usepackage[portuguese]{babel}
\usepackage{geometry}
\usepackage{amsmath}
\usepackage{graphicx}
\usepackage{hyperref}
\usepackage{listings}
\usepackage{xcolor}

\geometry{top=3cm, bottom=2cm, left=2.5cm, right=2.5cm}

\title{\textbf{SisterApp: Plataforma de Ecologia Computacional v3.8.0}\\Manual Técnico de Modelos Computacionais}
\author{José Pedro Trindade}
\date{\today}

\begin{document}

\maketitle
\tableofcontents
\newpage

\section{Introdução}
O \textbf{SisterApp} evoluiu de uma engine gráfica para uma plataforma científica robusta focada em ecologia computacional. A versão 3.8.0 consolida ferramentas de navegação, análise de paisagem e validação métrica.

\section{Modelo de Análise de Declividade (Slope Analysis)}
Este modelo substitui a anterior lógica abstrata de resiliência por uma abordagem quantitativa baseada na inclinação local do terreno.

% ... (Keep existing Slope sections) - I will use a larger replacement chunk or multiple replacing logic but the user tool only allows contiguous.
% I will skip modifying lines 27-171 in this call and focus on lines 14, 24, and 172.
% Actually, I can't skip in the middle of a replacement chunk if I want to replace title AND minimap which are far apart.
% Providing multiple replacement chunks.

% Chunk 1: Title & Intro (Lines 14-25)
% Chunk 2: Minimap Section (Lines 172-179)


\section{Modelo de Análise de Declividade (Slope Analysis)}
Este modelo substitui a anterior lógica abstrata de resiliência por uma abordagem quantitativa baseada na inclinação local do terreno.

\subsection{Cálculo de Inclinação (Percentual)}
A declividade é calculada como a razão entre a diferença de altura (rise) e a distância horizontal (run), expressa em porcentagem. Para um ponto no terreno, a inclinação $S_{\%}$ é dada por:

\[
S_{\%} = \frac{\sqrt{(\Delta x)^2 + (\Delta z)^2}}{\text{run}} \times 100
\]
Onde:
\begin{itemize}
    \item $\Delta x$ e $\Delta z$ são os gradientes de altura nas direções X e Z.
    \item A distância base (\textit{run}) é definida pela resolução do voxel (2 unidades).
\end{itemize}

Isto permite uma correlação direta com normas técnicas de engenharia civil.

\subsection{Classificação Topológica (5 Classes)}
O terreno é segmentado em classes configuráveis pelo usuário. Os limiares (thresholds) padrão são:

\begin{table}[h]
\centering
\begin{tabular}{|l|l|l|}
\hline
\textbf{Classe} & \textbf{Intervalo ($S_{\%}$)} & \textbf{Descrição} \\ \hline
Flat (Plano) & $0\% - 3.0\%$ & Áreas adequadas para infraestrutura. \\ \hline
Gentle Slope (Suave) & $3.0\% - 8.0\%$ & Áreas de transição suave. \\ \hline
Rolling (Ondulado) & $8.0\% - 20.0\%$ & Terreno ondulado, requer terraplanagem. \\ \hline
Steep Slope (Íngreme/Forte) & $20.0\% - 45.0\%$ & Encostas fortes, risco de erosão. \\ \hline
Mountain (Montanha) & $> 45.0\%$ & Áreas inacessíveis ou de preservação. \\ \hline
\end{tabular}
\caption{Classes de Declividade Padrão (v3.5)}
\end{table}

\subsection{Persistência e Configuração}
Diferente dos modelos anteriores, todas as configurações de declividade são \textbf{persistentes}. O sistema serializa os limiares definidos pelo usuário em um arquivo JSON (`prefs.json`), garantindo que os critérios de análise sejam mantidos entre sessões.

\section{Modelo de Vegetação (Suspenso)}
Na versão 3.4.0, a geração de vegetação foi temporariamente suspensa para permitir foco total na validação das camadas de análise topológica. O sistema de tipos de solo (Grass, Dirt, Stone) permanece ativo para feedback visual.

\section{Geração de Topologia (Terrain Models)}
É fundamental distinguir o \textbf{Gerador de Topologia} do \textbf{Analisador de Declividade}. O sistema mantém três perfis de geração baseados em ruído Perlin, que definem a geometria física do mundo:

\begin{itemize}
    \item \textbf{Rippled Flat:} Baixa frequência base, gera predominantemente classes \textit{Flat} e \textit{Gentle}.
    \item \textbf{Smooth Hills:} Frequência média, introduz áreas \textit{Rolling}.
    \item \textbf{Rolling Hills:} Alta amplitude, necessária para gerar áreas \textit{Steep} e \textit{Mountain} para validação.
\end{itemize}

O fluxo de processamento é:
\[ \text{Modelo (Geometria)} \rightarrow \text{Voxel Grid} \rightarrow \text{Slope Analysis (Classificação)} \]

\section{Configuração do Usuário}
Interface atualizada no menu \textit{Tools}:
\begin{description}
    \item[Slope Sliders:] Ajuste dos limites percentuais para cada classe.
    \item[Probe Tool:] Ferramenta de diagnóstico (clique esquerdo) mostra $S_{\%}$ exato.
    \item[Persistence:] Botões para salvar/carregar preferências manualmente.
\end{description}

\subsection{Variáveis de Controle Espacial}
O sistema permite o ajuste fino da topografia através de três variáveis principais:
\begin{enumerate}
    \item \textbf{Feature Size (Frequência):} Controla o tamanho horizontal das montanhas. Valores menores geram grandes maciços; valores maiores geram colinas frequentes.
    \item \textbf{Roughness (Persistência):} Controla a irregularidade da superfície. 
    \begin{itemize}
        \item Baixa ($<0.5$): Colinas suaves e dunas.
        \item Alta ($>0.5$): Terreno rochoso, escarpado e ruidoso.
    \end{itemize}
    \item \textbf{Amplitude:} A altura máxima vertical em metros.
    \item \textbf{Cell Size (Resolução):} A dimensão física de cada pixel da grade (em metros).
\end{enumerate}

\section{Modelo de Drenagem (D8 Flow)}
A partir da versão v3.6.0, o sistema substituiu o modelo estocástico de erosão por partículas por um algoritmo determinístico de drenagem D8 (Steepest Descent).

\subsection{Direção do Fluxo (Flow Direction)}
Para cada célula do grid de terreno, o algoritmo determina a direção de escoamento para um dos 8 vizinhos com maior \textbf{declividade} descendente (Steepest Slope).
\[
\text{Receiver} = \text{argmax}_{n \in \text{Neighbors}} \left( \frac{H_{\text{current}} - H_n}{\text{Distance}_n} \right)
\]
Onde $\text{Distance}_n$ é a distância física até o vizinho (Resolution para cardeais, $\text{Resolution} \times \sqrt{2}$ para diagonais).
Se o declive for $\leq 0$ para todos os vizinhos (mínimo local), a célula é um "sink" (sumidouro).

\subsection{Acumulação de Fluxo (Flow Accumulation)}
O fluxo é calculado iterativamente, ordenando as células por altura (decrescente). Cada célula transfere seu valor de fluxo acumulado para o seu vizinho receptor (Receiver), simulando a conservação de massa da água.
\[
F_{\text{receiver}} += F_{\text{upstream}}
\]
O resultado é um \textit{Flux Map} onde valores altos representam rios e canais principais.

\subsection{Visualização}
O shader utiliza o mapa de fluxo acumulado para renderizar recursos hídricos:
\begin{itemize}
    \item \textbf{Canais Principais:} Células com fluxo $F > 1.0$ (limite visual configurável) são coloridas em Cyan (0.0, 0.8, 1.0).
    \item \textbf{Continuidade:} O método D8 garante redes de drenagem dendríticas contínuas sem artefatos geométricos ("spots").
\end{itemize}

\section{Análise de Bacias Hidrográficas (Watershed Analysis)}
Introduzido na versão v3.6.3, este módulo permite a identificação e delimitação de bacias de drenagem baseadas na topologia D8.

\subsection{Segmentação Global}
O algoritmo de segmentação particiona todo o terreno em bacias distintas. O processo ocorre em duas etapas:
\begin{enumerate}
    \item \textbf{Identificação de Sinks}: Localização de todos os "sumidouros" (minimos locais ou bordas do mapa). Cada sink recebe um ID único.
    \item \textbf{Propagação Upstream (BFS)}: Um algoritmo de busca em largura (Breadth-First Search) percorre a rede de fluxo no sentido inverso (de jusante para montante), atribuindo o ID do sink a todas as células constituintes de sua área de contribuição.
\end{enumerate}

\subsection{2. Delineação Interativa}
Permite ao usuário consultar a bacia de contribuição de um ponto arbitrário $P(x,y)$. O sistema rastreia recursivamente todos os vizinhos que fluem para $P$, gerando uma máscara binária instantânea da área de captação a montante.

\subsection{3. Visualização de Contornos}
O usuário pode habilitar a opção "Show Contours" na interface. O sistema utiliza a derivada parcial do ID da bacia (via shader \texttt{fwidth}) para detectar arestas onde o ID muda, desenhando uma linha escura de 1 pixel sobre os limites das bacias para melhor distinção visual.

\section{Métricas Eco-Hidrológicas}
O Relatório Hidrológico foi expandido para incluir indicadores funcionais derivados da topografia:

\subsection{Índice Topográfico de Umidade (TWI)}
\[ TWI = \ln \left( \frac{A}{\tan \beta} \right) \]
Onde $A$ é a área de contribuição específica (fluxo) e $\tan \beta$ é a declividade local. O TWI estima zonas de saturação do solo. O sistema reporta a porcentagem da área com $TWI > 8.0$ como proxy para zonas úmidas.

\subsection{Densidade de Drenagem ($D_d$)}
\[ D_d = \frac{L_{total}}{Area_{total}} \]
Calculado como a razão entre células classificadas como "rio" (Fluxo > 100) e o total de células. Indica a permeabilidade e dissecação do relevo.

\subsection{Estatísticas por Bacia (Basin-Level Metrics)}
O sistema agora agrega métricas de elevação, declividade, TWI e densidade de drenagem individualmente para as 3 maiores bacias identificadas, permitindo uma análise comparativa da resposta hidrológica de diferentes sub-regiões do modelo.

\section{Resolução Espacial Variável (V3.6.5)}
Para atender à necessidade de maior definição nos limites de bacias e redes de drenagem, foi introduzido o controle de \textbf{Cell Size (Resolução)}.

\subsection{Definição de Escala}
O usuário pode ajustar o tamanho métrico de cada célula (pixel) da grade de simulação:
\begin{itemize}
    \item \textbf{1.0 m (Padrão):} Equilíbrio entre cobertura de área e detalhe.
    \item \textbf{$< 1.0$ m (Alta Resolução):} Aumenta a densidade de vértices por unidade de área. Ideal para suavizar limites de bacias e detalhar canais de drenagem, reduzindo o efeito de "pixelização" (aliasing geométrico).
    \item \textbf{$> 1.0$ m (Baixa Resolução):} Permite cobrir grandes extents geográficos com menor custo computacional.
\end{itemize}

O sistema ajusta automaticamente a visualização e a lógica de interação (raycasting) para manter a coerência espacial independentemente da escala escolhida.

\section{Módulo de Análise de Solos (V3.7.3)}
O sistema inclui agora uma camada de pedologia probabilística baseada na declividade, conforme a tabela de relação Relevo-Solo definida pelo usuário.

\subsection{Metodologia: Ruído Coerente e Métricas de Paisagem (v3.8.0).}
\subsection{Minimap e Navegação Interativa}
A versão 3.8.0 introduz um Minimapa e controles de câmera aprimorados para facilitar a navegação e a compreensão espacial.
\begin{itemize}
    \item \textbf{Visualização Top-Down}: Renderiza o mapa de solos e relevo com neblina de guerra (Fog of War) simulada pela distância.
    \item \textbf{Símbolos (Alegorias)}: Um algoritmo de detecção de picos identifica máximos locais na topografia e desenha pequenos triângulos brancos, fornecendo referências visuais "game-like" para orientação.
    \item \textbf{Nível da Água (Water Level)}: Visualização configurável de zonas submersas (Azul), permitindo identificar depressões e lagos mesmo antes da simulação hidrológica.
    \item \textbf{Controles}:
    \begin{itemize}
        \item \textbf{Zoom}: Roda do mouse ajusta o Campo de Visão (FOV) no modo voo livre.
        \item \textbf{Minimap Zoom/Pan}: Roda do mouse e botão do meio dentro da janela do minimapa.
        \item \textbf{Teleporte}: Clique com botão esquerdo no minimapa para viagem rápida.
    \end{itemize}
\end{itemize}
Para reproduzir os padrões espaciais descritos pelos índices de Ecologia da Paisagem (LSI, CF, RCC), o sistema substituiu a distribuição aleatória simples por um algoritmo de \textbf{Competição de Padrões} baseado em ruído procedural (Perlin/Simplex).

Cada tipo de solo possui um "perfil de ruído" configurado para mimetizar suas métricas (Ver Tabela \ref{tab:manchas_solo_farina}):
\begin{itemize}
    \item \textbf{Domain Warping (Distorção):} Simula o LSI. Solos com alto LSI sofrem forte distorção de coordenadas, criando bordas complexas.
    \item \textbf{Frequência e Rugosidade:} Simulam o CF. Solos com alto CF utilizam mais oitavas de ruído fractal.
    \item \textbf{Anisotropia (Estiramento):} Simula o RCC. Solos com baixo RCC são esticados em um eixo para criar formas alongadas.
\end{itemize}

O solo final em cada pixel é determinado por uma competição onde o tipo com maior "força" de padrão local vence (dentre os candidatos válidos para a declividade).

\subsection{Classes e Cores}
\begin{itemize}
    \item \textbf{Plano (0-3\%):} Hidromórfico (Teal), B Textural (Laranja), Argila Expansiva (Roxo).
    \item \textbf{Suave (3-8\%):} B Textural, Bem Desenvolvido (Terracota), Argila Expansiva.
    \item \textbf{Ondulado (8-20\%):} B Textural, Argila Expansiva.
    \item \textbf{Forte (20-45\%):} B Textural, Solo Raso (Amarelo).
    \item \textbf{Montanhoso (45-75\%):} Solo Raso (Amarelo).
    \item \textbf{Escarpado ($>75\%$):} Afloramento Rochoso (Cinza).
\end{itemize}


\begin{table}[htbp]
	\centering
	\caption{Descritores de estrutura espacial das manchas de solo segundo métricas da Ecologia da Paisagem (Farina)}
	\label{tab:manchas_solo_farina}
	\begin{tabular}{lccc}
		\hline
		\textbf{Tipo de Solo} & 
		\textbf{LSI} & 
		\textbf{CF} & 
		\textbf{RCC} \\
		\hline
		Solo Raso & 
		5434.91 & 
		2.49 & 
		0.66 \\
		
		Bem Desenvolvido & 
		2508.07 & 
		2.36 & 
		0.68 \\
		
		Hidromórfico & 
		3272.30 & 
		2.27 & 
		0.65 \\
		
		Argila Expansiva & 
		1827.24 & 
		2.84 & 
		0.64 \\
		
		B--Textural & 
		2766.09 & 
		3.36 & 
		0.66 \\
		\hline
	\end{tabular}
	\begin{flushleft}
		\footnotesize
		\textbf{Nota:} 
		LSI = Índice de Forma da Paisagem (\textit{Landscape Shape Index}), indicador da complexidade geométrica média das manchas; 
		CF = Complexidade da Forma, expressando irregularidade e alongamento das manchas; 
		RCC = Coeficiente de Circularidade Relativa, variando de 0 (formas alongadas ou irregulares) a 1 (formas altamente compactas).
	\end{flushleft}
\end{table}

\subsection*{Descritores de forma das manchas}

A estrutura espacial das manchas de solo foi caracterizada por descritores clássicos da Ecologia da Paisagem, conforme proposto por Farina (1998, 2006), os quais permitem avaliar a complexidade geométrica, a irregularidade das bordas e o grau de compactação das manchas na paisagem. Foram utilizados o Índice de Forma da Paisagem (LSI), a Complexidade da Forma (CF) e o Coeficiente de Circularidade Relativa (RCC), descritos a seguir.

\paragraph{Índice de Forma da Paisagem (LSI)}
O LSI expressa a complexidade geométrica das manchas a partir da relação entre perímetro e área, sendo definido por:
\[
LSI = \frac{P}{2\sqrt{\pi A}}
\]
em que $P$ corresponde ao perímetro da mancha e $A$ à sua área. Valores de LSI próximos de 1 indicam manchas com formas simples e compactas, enquanto valores mais elevados refletem maior irregularidade e desenvolvimento de bordas.

\paragraph{Complexidade da Forma (CF)}
A Complexidade da Forma representa o grau de irregularidade geométrica das manchas, considerando o aumento relativo do perímetro em relação à área:
\[
CF = \frac{P}{A}
\]
em que $P$ é o perímetro e $A$ a área da mancha. Valores mais elevados de CF indicam formas mais alongadas, recortadas ou dendríticas.

\paragraph{Coeficiente de Circularidade Relativa (RCC)}
O Coeficiente de Circularidade Relativa avalia o quão próxima a forma da mancha está de um círculo perfeito, sendo calculado por:
\[
RCC = \frac{4\pi A}{P^{2}}
\]
em que $A$ corresponde à área da mancha e $P$ ao seu perímetro. O RCC varia entre 0 e 1, sendo valores próximos de 1 indicativos de manchas altamente compactas e valores menores associados a formas alongadas ou fragmentadas.


\end{document}
