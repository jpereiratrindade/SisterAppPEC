\documentclass[a4paper,12pt]{article}
\usepackage[utf8]{inputenc}
\usepackage[T1]{fontenc}
\usepackage[portuguese]{babel}
\usepackage{geometry}
\usepackage{amsmath}
\usepackage{graphicx}
\usepackage{hyperref}
\usepackage{listings}
\usepackage{xcolor}

\geometry{top=3cm, bottom=2cm, left=2.5cm, right=2.5cm}

\title{\textbf{SisterApp Engine v3.3.0}\\Manual Técnico de Modelos Computacionais}
\author{José Pedro Trindade}
\date{\today}

\begin{document}

\maketitle
\tableofcontents
\newpage

\section{Introdução}
O \textbf{SisterApp Engine} integra sistemas avançados de análise topológica e persistência de dados. A versão 3.4.0 introduz um novo modelo focado na \textbf{Análise de Declividade (Slope Analysis)}, permitindo classificações precisas do terreno para aplicações de engenharia e planejamento.

\section{Modelo de Análise de Declividade (Slope Analysis)}
Este modelo substitui a anterior lógica abstrata de resiliência por uma abordagem quantitativa baseada na inclinação local do terreno.

\subsection{1. Cálculo de Inclinação (Percentual)}
A declividade é calculada como a razão entre a diferença de altura (rise) e a distância horizontal (run), expressa em porcentagem. Para um ponto no terreno, a inclinação $S_{\%}$ é dada por:

\[
S_{\%} = \frac{\sqrt{(\Delta x)^2 + (\Delta z)^2}}{\text{run}} \times 100
\]
Onde:
\begin{itemize}
    \item $\Delta x$ e $\Delta z$ são os gradientes de altura nas direções X e Z.
    \item A distância base (\textit{run}) é definida pela resolução do voxel (2 unidades).
\end{itemize}

Isto permite uma correlação direta com normas técnicas de engenharia civil.

\subsection{2. Classificação Topológica}
O terreno é segmentado em classes configuráveis pelo usuário. Os limiares (thresholds) padrão são:

\begin{table}[h]
\centering
\begin{tabular}{|l|l|l|}
\hline
\textbf{Classe} & \textbf{Intervalo ($S_{\%}$)} & \textbf{Descrição} \\ \hline
Flat (Plano) & $0\% - 3\%$ & Áreas adequadas para infraestrutura. \\ \hline
Gentle Slope (Suave) & $3\% - 10\%$ & Áreas de transição. \\ \hline
Steep Slope (Íngreme) & $10\% - 45\%$ & Terreno acidentado, requer contenção. \\ \hline
Mountain (Montanha) & $> 45\%$ & Áreas inacessíveis ou de preservação. \\ \hline
\end{tabular}
\caption{Classes de Declividade Padrão (v3.4.0)}
\end{table}

\subsection{3. Persistência e Configuração}
Diferente dos modelos anteriores, todas as configurações de declividade são \textbf{persistentes}. O sistema serializa os limiares definidos pelo usuário em um arquivo JSON (`prefs.json`), garantindo que os critérios de análise sejam mantidos entre sessões.

\section{Modelo de Vegetação (Suspenso)}
Na versão 3.4.0, a geração de vegetação foi temporariamente suspensa para permitir foco total na validação das camadas de análise topológica. O sistema de tipos de solo (Grass, Dirt, Stone) permanece ativo para feedback visual.

\section{Configuração do Usuário}
Interface atualizada no menu \textit{Tools}:
\begin{description}
    \item[Slope Sliders:] Ajuste dos limites percentuais para cada classe.
    \item[Probe Tool:] Ferramenta de diagnóstico (clique esquerdo) mostra $S_{\%}$ exato.
    \item[Persistence:] Botões para salvar/carregar preferências manualmente.
\end{description}

\end{document}
