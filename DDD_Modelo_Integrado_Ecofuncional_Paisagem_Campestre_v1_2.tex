\documentclass[
12pt,
oneside,
a4paper,
brazil
]{article}

% -------------------------------------------------
% Pacotes
% -------------------------------------------------
\usepackage[utf8]{inputenc}
\usepackage[T1]{fontenc}
\usepackage[brazil]{babel}
\usepackage{lmodern}
\usepackage{microtype}
\usepackage{geometry}
\usepackage{setspace}
\usepackage{indentfirst}
\usepackage{amsmath, amssymb}
\usepackage{hyperref}
\usepackage{enumitem}

% -------------------------------------------------
% Configurações
% -------------------------------------------------
\geometry{top=2.5cm,bottom=2.5cm,left=3cm,right=2.5cm}
\onehalfspacing
\setlength{\parindent}{1.25cm}

% -------------------------------------------------
% Título
% -------------------------------------------------
\title{\textbf{DDD — Modelo Integrado Ecofuncional da Paisagem Campestre}\\
	\large Fundamentos conceituais e arquiteturais para simulação ecofuncional em larga escala}

\author{
	José Pedro Trindade\\
	\small Plataforma de Ecologia Computacional -- SisTerApp
}

\date{Dezembro de 2025}

% -------------------------------------------------
\begin{document}
	\maketitle
	
	% =================================================
	\section{Domínio}
	% =================================================
	
	\textbf{Domínio:}  
	Modelagem integrada da paisagem campestre como um sistema ecofuncional complexo,
	espacialmente acoplado e temporalmente explícito, no qual componentes físicos,
	biológicos e hidrológicos interagem para produzir funções ecológicas mensuráveis
	e propriedades emergentes.
	
	\textbf{Escopo do domínio:}
	\begin{itemize}
		\item Tipos de solo e sua organização espacial;
		\item Relevo, hidrologia superficial e conectividade topográfica;
		\item Estrutura, dinâmica e memória da vegetação campestre;
		\item Processos ecofísicos e ecobiológicos;
		\item Funções ecológicas fundamentais;
		\item Propriedades emergentes do sistema.
	\end{itemize}
	
	\textbf{Princípio estruturante:}  
	Nenhuma propriedade sistêmica é modelada diretamente. Todas emergem da interação
	entre estados, processos, funções e regimes de distúrbio explicitamente representados.
	
	% =================================================
	\section{Subdomínios}
	% =================================================
	
	\begin{itemize}
		\item \textbf{Ambiente Físico} (Supporting)
		\item \textbf{Vegetação Campestre} (Core)
		\item \textbf{Hidrologia Superficial} (Core)
		\item \textbf{Processos Ecofuncionais} (Core)
		\item \textbf{Funções Ecofuncionais} (Core)
		\item \textbf{Propriedades Emergentes} (Generic)
	\end{itemize}
	
	% =================================================
	\section{Agregado Central e Considerações Arquiteturais}
	% =================================================
	
	\subsection{LandscapeCell (Raiz de Agregado Lógica)}
	
	A \textit{LandscapeCell} representa a unidade lógica mínima da paisagem no domínio
	conceitual.
	
	\textbf{Identidade}
	\begin{itemize}
		\item CellID
	\end{itemize}
	
	\textbf{Responsabilidades Conceituais}
	\begin{itemize}
		\item Representar um ponto da paisagem ecofuncional;
		\item Referenciar estados ambientais e biológicos locais;
		\item Servir como base para processos, funções e métricas emergentes.
	\end{itemize}
	
	\textbf{Nota Arquitetural:}  
	Embora conceitualmente tratada como uma entidade integrada no DDD, a
	\textit{LandscapeCell} pode ser implementada computacionalmente como um identificador
	(index), com seus estados distribuídos em componentes independentes (ex.: SoilState,
	VegetationState) organizados em estruturas orientadas a dados (SoA / ECS), visando
	processamento paralelo e eficiência de cache.
	
	% =================================================
	\section{Estados de Domínio}
	% =================================================
	
	\subsection{SoilState}
	
	\begin{itemize}
		\item soilType
		\item soilDepth
		\item surfaceExposure (fração de solo descoberto)
		\item infiltrationCapacity
		\item compactionLevel
		\item propaguleBank (potencial regenerativo)
	\end{itemize}
	
	\subsection{TerrainState}
	
	\begin{itemize}
		\item slope
		\item curvature
		\item topographicPosition
	\end{itemize}
	
	\subsection{VegetationState}
	
	\begin{itemize}
		\item Coverage$_{EI}$
		\item Coverage$_{ES}$
		\item Vigor$_{EI}$
		\item Vigor$_{ES}$
		\item structuralHeterogeneity
		\item propaguleBank (sementes, gemas, memória biológica)
	\end{itemize}
	
	\subsection{HydrologicalState}
	
	\begin{itemize}
		\item flowAccumulation
		\item runoffPotential
		\item laminarErosionRisk
	\end{itemize}
	
	% =================================================
	\section{Vizinhança Espacial e Interações de Borda}
	% =================================================
	
	A paisagem é modelada como um sistema espacialmente acoplado.
	
	\textbf{Definição de Vizinhança:}
	\begin{itemize}
		\item Vizinhança de Moore (8 células adjacentes) para processos difusivos;
		\item Vizinhança direcionada (ex.: D8) para escoamento hidrológico.
	\end{itemize}
	
	As interações de borda incluem:
	\begin{itemize}
		\item Fluxo de água entre células;
		\item Propagação de distúrbios (fogo, erosão);
		\item Recolonização vegetal lateral.
	\end{itemize}
	
	% =================================================
	\section{Processos Ecofísicos e Ecobiológicos}
	% =================================================
	
	Os processos ecofuncionais são formalizados como \textbf{Serviços de Domínio}, pois
	operam sobre conjuntos de células e suas interações espaciais.
	
	\textbf{Exemplos de Serviços de Domínio:}
	\begin{itemize}
		\item GrowthService (crescimento e recuperação vegetal);
		\item RunoffService (escoamento superficial e fluxo hidrológico);
		\item ErosionService (perda progressiva de solo);
		\item PropagationService (dispersão e recolonização);
		\item CompactionService (pisoteio e selamento superficial).
	\end{itemize}
	
	Esses serviços não pertencem a uma única célula isolada, mas à dinâmica do sistema
	como um todo.
	
	% =================================================
	\section{Tempo e Loop de Atualização}
	% =================================================
	
	O sistema opera em tempo discreto, organizado em passos de simulação ($\Delta t$).
	
	\textbf{Loop de Atualização Conceitual:}
	\begin{enumerate}
		\item Aplicação de regimes de distúrbio;
		\item Execução dos processos ecofuncionais;
		\item Atualização dos estados de domínio;
		\item Cálculo das funções ecofuncionais;
		\item Derivação das propriedades emergentes.
	\end{enumerate}
	
	O tempo é tratado como entidade implícita do domínio, sendo essencial para a
	interpretação de resiliência e trajetórias ecofuncionais.
	
	% =================================================
	\section{Regimes de Distúrbio}
	% =================================================
	
	Os regimes de distúrbio são agentes estruturantes da paisagem.
	
	\begin{itemize}
		\item tipo (fogo, pastejo, pisoteio, seca);
		\item frequência;
		\item magnitude;
		\item seletividade funcional;
		\item escala espacial.
	\end{itemize}
	
	Eles modulam trajetórias do sistema, não sendo considerados ruído aleatório.
	
	% =================================================
	\section{Funções Ecofuncionais}
	% =================================================
	
	As funções ecofuncionais traduzem estados e processos em capacidades ecológicas.
	
	\begin{itemize}
		\item Infiltração de água;
		\item Proteção do solo;
		\item Produção primária;
		\item Estabilidade estrutural da vegetação.
	\end{itemize}
	
	\[
	F_{prot} = f(C_{EI}, C_{ES}, H_{estrutural})
	\]
	
	% =================================================
	\section{Diversidade como Função Derivada}
	% =================================================
	
	\begin{itemize}
		\item Diversidade específica (proxy estrutural-funcional);
		\item Diversidade funcional (variedade de respostas);
		\item Diversidade estrutural (heterogeneidade espacial).
	\end{itemize}
	
	% =================================================
	\section{Propriedades Emergentes}
	% =================================================
	
	\subsection{Resiliência Ecofuncional}
	
	\[
	R = g(F_{prot}, F_{inf}, F_{prod}, D, T)
	\]
	
	A resiliência é uma propriedade sistêmica derivada, não um estado primário.
	
	% =================================================
	\section{Invariantes do Sistema}
	% =================================================
	
	\begin{enumerate}
		\item Solo descoberto elevado implica aumento de erosão;
		\item A perda funcional precede o colapso estrutural;
		\item Regeneração depende de memória biológica local;
		\item Propriedades emergentes não retroagem diretamente sobre estados.
	\end{enumerate}
	
	% =================================================
	\section{Linguagem Ubíqua do Domínio}
	% =================================================
	
	\begin{itemize}
		\item LandscapeCell
		\item Serviço de Domínio
		\item Processo ecofuncional
		\item Regime de distúrbio
		\item Banco de propágulos
		\item Função ecofuncional
		\item Resiliência ecofuncional
		\item Mancha campestre
	\end{itemize}
	
	% =================================================
	\section{Síntese}
	% =================================================
	
	Este documento consolida um modelo ecofuncional integrado da paisagem campestre,
	alinhando fundamentos ecológicos, clareza ontológica e viabilidade computacional
	para simulações em larga escala e aplicações científicas e extensionistas.
	
\end{document}
